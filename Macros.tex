% A macro \preambulo{〈preâmbulo do documento〉} é utilizada para armazenar o\preambulo
% \imprimirpreambulo preâmbulo do documento. O preâmbulo é o texto impresso na Folha de rosto e na
% Folha de aprovação. Ele deve conter o tipo do documento, o objetivo, o nome da
% instituição e a área de concentração. O conteúdo armazenado é impresso por meio
% da macro \imprimirpreambulo.

% \orientador[〈rótulo〉]{〈nome do(s) orientador(es)〉} -- \imprimirorientador -- \imprimirorientadorRotulo
% \coorientador[〈rótulo〉]{〈nome do(s) coorientador(es)〉 -- \imprimircoorientador -- \imprimircoorientadorRotulo
% \tipotrabalho{〈tipo do trabalho〉} -- \imprimirtipotrabalho

% TABELAS E FIGURAS
% CAPTION
% TABELA/FIGURA
% FONTE

% REFERENCIAS
% acento bibtex
% à á ã \‘a \’a \~a
% í {\’\i}
% ç {\c c}

% Macros.tex — macros e comandos reutilizáveis do projeto

% Pacotes usados no projeto já estão em pacotes.sty, então evite carregar pacotes aqui.
% Se precisar de um pacote extra, adicione em pacotes.sty ou me peça para consolidar.

% ---------------------------
% Atalhos de texto
% ---------------------------
\newcommand{\SO}{Sistemas Operacionais}        % uso: \SO
\newcommand{\UVV}{Universidade de Vila Velha}  % uso: \UVV

% ---------------------------
% Frases e elementos repetidos
% ---------------------------
\newcommand{\discipline}{Disciplina de Sistemas Operacionais}
\newcommand{\semester}{2\textordfeminine\ Bimestre}

% ---------------------------
% Unidades com siunitx (pacote já carregado)
% ---------------------------
\newcommand{\kbps}[1]{\SI{#1}{\kilo\bit\per\second}}
% exemplo: \kbps{100} -> 100 kbit/s

% ---------------------------
% Macros matemáticas simples
% ---------------------------
\newcommand{\vect}[1]{\mathbf{#1}} % vetor em negrito: \vect{v}

% ---------------------------
% Abreviações e glossário (use glossaries se preferir)
% ---------------------------
% Se quiser que o glossário seja gerenciado, use \newglossaryentry no preâmbulo.
% Exemplo de atalho textual:
\newcommand{\eg}{\emph{e.g.}\ } % cuidado com espaçamento; teste em texto

% ---------------------------
% Comandos para referências
% ---------------------------
\newcommand{\figref}[1]{Figura~\ref{#1}}
\newcommand{\secref}[1]{Seção~\ref{#1}}

% ---------------------------
% Mensagens de TODO/NOTA (visíveis no PDF)
% ---------------------------
\newcommand{\nota}[1]{\textbf{[NOTA: #1]}}

% ---------------------------
% Ajustes para BibTeX/acento (se precisar)
% ---------------------------
% Exemplos de mapeamento (use somente se tiver problemas com acentos no .bib)
\newcommand{\Acutea}{\'a}  % se necessário mapeamento para o .bib

% ---------------------------
% Fim — documente cada macro acima para futuros autores
% ---------------------------