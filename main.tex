\documentclass[
12pt, a4paper, twoside, chapter=TITLE, subsection=TITLE, section=TITLE, subsubsection=TITLE, subsubsubsection=TITLE, english, german, brazil
]{abntex2}

\usepackage[utf8]{inputenc}
\usepackage[brazilian,hyperpageref]{backref}
\usepackage{pacotes}

\titulo{Trabalho Sistemas Operacionais \\ 2° Bimestre}
\tituloestrangeiro{--texto do título estrangeiro--}
\autor
{
Endrews Elias de Oliveira Prado \\
Jhuan da Matta Julião de Oliveira \\
Lucas Mattos Canela \\
Robert Valadão da Silva \\
Victor Griffo de Andrade Faria
}
\orientador{Jean-Remi}
\coorientador{}
\data{\today}
\instituicao{
    Universidade de Vila Velha (UVV) \par
    Graduação em Ciência da Computação
}
\tipotrabalho{Trabalho Acadêmico}
\local{Vila Velha, ES}
%\preambulo{--Trabalho acadêmico como requisito parcial para aprovação na disciplina Sistemas Operacionais--.}

% informações do PDF
\makeatletter
\hypersetup{
    pdftitle={\@title},
    pdfauthor={\@author},
    pdfsubject={\imprimirpreambulo},
    pdfkeywords={PALAVRAS}{CHAVE}{EM}{PORTUGUES},
    pdfcreator={LaTeX with abnTeX2},
    colorlinks=true,
    linkcolor=blue,
    citecolor=blue,
    urlcolor=blue
    }
\makeatother
\setlength{\absparsep}{18pt} % ajusta o espaçamento dos parágrafos do resumo


% ---
% compila o indice
% ---
\makeindex
% ---
 
% ---
% GLOSSARIO
% ---
\makeglossaries
 
% ---
% entradas do glossario
% ---
%  \newglossaryentry{pai}{
%                 name={pai},
%                 plural={pai},
%                 description={este é uma entrada pai, que possui outras
%                 subentradas.} }

%  \newglossaryentry{componente}{
%                 name={componente},
%                 plural={componentes},
%                 parent=pai,
%                 description={descriação da entrada componente.} }
 
%  \newglossaryentry{filho}{
%                 name={filho},
%                 plural={filhos},
%                 parent=pai,
%                 description={isto é uma entrada filha da entrada de nome
%                 \gls{pai}. Trata-se de uma entrada irmã da entrada
%                 \gls{componente}.} }
 
% \newglossaryentry{equilibrio}{
%                 name={equilíbrio da configuração},
%                 see=[veja também]{componente},
%                 description={consistência entre os \glspl{componente}}
%                 }

% \newglossaryentry{latex}{
%                 name={LaTeX},
%                 description={ferramenta de computador para autoria de
%                 documentos criada por D. E. Knuth} }

% \newglossaryentry{abntex2}{
%                 name={abnTeX2},
%                 see=latex,
%                 description={suíte para LaTeX que atende os requisitos das
%                 normas da ABNT para elaboração de documentos técnicos e científicos brasileiros} }
% % ---

% % ---
% % Exemplo de configurações do glossairo
% \renewcommand*{\glsseeformat}[3][\seename]{\textit{#1}  
%  \glsseelist{#2}}
% % ---


\begin{document}
\pretextual 
\imprimircapa
\imprimirfolhaderosto[]* % talvez dê merda na numeração das páginas
\makeatother\cleardoublepage

% ---
% Epígrafe
% ---
\begin{epigrafe}
    \vspace*{\fill}
	\begin{flushright}
		\textit
        {
            "Grandes poderes trazem \\
            grandes responsabilidades." \\
            (Homem-Aranha)
        }
	\end{flushright}
\end{epigrafe}
% ---

\begin{resumo}[Resumo] 
    ------- Resumo em português. -----  \\
    \vspace{\onelineskip}
    \noindent
    \textbf{Palavras-chave}: latex. abntex. editoração de texto.
\end{resumo}

\begin{resumo}[Abstract]
    \begin{otherlanguage*}{english}
        ------ abstract. ---- \\
        \vspace{\onelineskip}
        \noindent
        \textbf{Keywords}: latex. abntex. publication de textes.
    \end{otherlanguage*}
\end{resumo}


\pdfbookmark[0]{\listfigurename}{lof}
\listoffigures*
\cleardoublepage


% \pdfbookmark[0]{\listtablename}{lot}
% \listoftables*
% \cleardoublepage


\begin{siglas}
    \item[sigla] significado da sigla;
\end{siglas}


% \begin{simbolos}
%     \item[simbolo] descrição;
% \end{simbolos}


\pdfbookmark[0]{\contentsname}{toc}
\tableofcontents*
\cleardoublepage


\textual % a partir daqui são os elementos textuais.
% \include{XXX} % adicionar os .tex da pasta inputs

\postextual % a partir daqui são os elementos pós-textuais.
\bibliography{ref} % O alerta amarelo ("Empty `thebibliography' environment on input line 8.") é porque não há citações ainda


\part{Discos}

\chapter{Processo de Boot}

\chapter{Memória Secundária}

\chapter{Partições}

\chapter{Sistema de Arquivos}

\part{Permissões}

\chapter{Grupos, Usuários, Poderes e Permissões}

\chapter{Gerenciamento de Processos}

\chapter{Gerenciamento de Memória Primária}

\addcontentsline{toc}{chapter}{\glossaryname}
\printglossaries

% \chapter{APÊNDICE}
% \chapter{anexo}
\end{document}
