\PassOptionsToPackage{backref=page}{hyperref}
\documentclass
[
    12pt, a4paper, twoside, chapter=TITLE, subsection=TITLE, section=TITLE, subsubsection=TITLE, subsubsubsection=TITLE, english, brazilian
]{abntex2}

% Configurar hyperref antes de ser carregado pela classe/pacotes
    
\usepackage[utf8]{inputenc}
    
% backref será configurado via hyperref (em pacotes.sty) para evitar conflitos
% Removido carregamento explícito do pacote `backref` para evitar conflito de opções;
% configure o backref via opções do hyperref (por exemplo, em pacotes.sty ou com \PassOptionsToPackage).

\usepackage{pacotes}

% Compatibilidade com memoir: evitar erro "Font command \tt is not supported"
% Alguns estilos/arquivos antigos ainda usam \tt; redirecionamos para \ttfamily
\let\tt\ttfamily

% Evitar erro "\verb illegal in argument" em strings de PDF (bookmarks etc.)
% Sanitiza comandos verbatim quando escritos em metadados pelo hyperref
\pdfstringdefDisableCommands{%%
    \def\verb#1{#1}%%
    \def\texttt#1{#1}%%
    \def\url#1{#1}%%
}

\renewcommand{\backrefpagesname}{Citado na(s) página(s):~}
\renewcommand{\backref}{}
\renewcommand{\backrefalt}[4]{%
    \ifcase #1 %
        Nenhuma citação no texto.%
    \or
        Citado na página #2.%
    \else
        Citado nas páginas #2.%
    \fi} 

\titulo{Conceitos de Sistemas Operacionais \\ 2° Bimestre}
%\tituloestrangeiro{--texto do título estrangeiro--}

\autor
{
Endrews Elias de Oliveira Prado \\
Jhuan da Matta Julião de Oliveira \\
Lucas Mattos Canela \\
Pedro Henrique de Oliveira Vieira Martins \\
Robert Valadão da Silva \\
Victor Griffo de Andrade Faria
}
\orientador{Jean-Remi}
%\coorientador{}
\data{\today}
\instituicao{
    Universidade de Vila Velha (UVV) \par
    Graduação em Ciência da Computação
}

\tipotrabalho{Trabalho Acadêmico}
\local{Vila Velha, ES}

%\preambulo{--Trabalho acadêmico como requisito parcial para aprovação na disciplina Sistemas Operacionais--.}
\pagestyle{abntheadings}
\aliaspagestyle{chapter}{abntheadings}


% informações do PDF
\makeatletter
\hypersetup{
    pdftitle={\@title},
    pdfauthor={\@author},
    pdfsubject={\imprimirpreambulo},
    pdfkeywords={PALAVRAS, CHAVE, EM, PORTUGUES},
    pdfcreator={LaTeX with abnTeX2},
    colorlinks=true,
    linkcolor=blue,
    citecolor=blue,
    urlcolor=blue,
}
\makeatother
\setlength{\absparsep}{18pt} % ajusta o espaçamento dos parágrafos do resumo


% ---
% compila o indice
% ---
\makeindex
% ---
 
% ---
% GLOSSARIO
% ---
\makeglossaries
 
% ---
% entradas do glossario
% ---
\newglossaryentry{latex}{name={LaTeX},description={ferramenta de computador para autoria dedocumentos criada por D. E. Knuth} }

\newglossaryentry{abntex2}{name={abnTeX2},see=latex,description={suíte para LaTeX que atende os requisitos dasnormas da ABNT para elaboração de documentos técnicos e científicos brasileiros} }

% --- Termos técnicos adicionados ao glossário ---
\newglossaryentry{boot}{name={boot}, description={processo de inicialização que prepara o kernel para execução}}
\newglossaryentry{bios}{name={BIOS}, description={firmware legado em ROM que realiza POST e inicia o boot}, see={boot}}
\newglossaryentry{uefi}{name={UEFI}, description={padrão moderno de firmware com NVRAM e partição EFI}, see={boot}}
\newglossaryentry{esp}{name={ESP}, description={EFI System Partition: partição FAT usada por carregadores UEFI}, see={uefi}}
\newglossaryentry{mbr}{name={MBR}, description={Master Boot Record: setor inicial com código de boot e tabela simples de partições}, see={gpt}}
\newglossaryentry{gpt}{name={GPT}, description={GUID Partition Table: esquema com endereçamento 64 bits e redundância}, see={mbr}}
\newglossaryentry{bootloader}{name={bootloader}, description={programa (ex: GRUB) que carrega e transfere controle ao kernel}, see={boot}}
\newglossaryentry{initramfs}{name={initramfs}, description={sistema de arquivos temporário usado na fase inicial do kernel}, see={boot}}
\newglossaryentry{hdd}{name={HDD}, description={disco magnético com trilhas, setores e cilindros}, see={ssd}}
\newglossaryentry{ssd}{name={SSD}, description={unidade de estado sólido baseada em memória flash}, see={nvme}}
\newglossaryentry{nvme}{name={NVMe}, description={protocolo otimizado para acesso a SSD via PCIe}, see={ssd}}
\newglossaryentry{chs}{name={CHS}, description={Cylinder-Head-Sector: antigo esquema físico de endereçamento}, see={lba}}
\newglossaryentry{lba}{name={LBA}, description={Logical Block Addressing: endereçamento linear de blocos}, see={chs}}
\newglossaryentry{inode}{name={inode}, description={estrutura de metadados de arquivo (permissões, apontadores)}}
\newglossaryentry{bloco}{name={bloco}, description={unidade mínima de alocação em sistemas de arquivos}, see={inode}}
\newglossaryentry{journaling}{name={journaling}, description={registro de transações para recuperação consistente}, see={inode}}
\newglossaryentry{cow}{name={Copy-on-Write}, description={adiar cópia até modificação garantindo integridade}, see={inode}}
\newglossaryentry{hardlink}{name={hard link}, description={nome adicional para o mesmo inode; mantém dados se o outro nome é removido}}
\newglossaryentry{softlink}{name={link simbólico}, description={arquivo que referencia caminho alvo; quebra se alvo removido}}
\newglossaryentry{uid}{name={UID}, description={identificador único de usuário}, see={gid}}
\newglossaryentry{gid}{name={GID}, description={identificador de grupo agregando usuários}, see={uid}}
\newglossaryentry{sudo}{name={sudo}, description={mecanismo para executar comandos com privilégios elevados}, see={root}}
\newglossaryentry{root}{name={root}, description={usuário administrativo com UID 0}, see={sudo}}
\newglossaryentry{pid}{name={PID}, description={identificador único de processo}, see={pcb}}
\newglossaryentry{pcb}{name={PCB}, description={Process Control Block: contexto do processo}, see={pid}}
\newglossaryentry{contexto}{name={troca de contexto}, description={alternância da CPU entre processos salvando/restaurando estado}, see={pid}}
\newglossaryentry{mmu}{name={MMU}, description={unidade que traduz endereços virtuais em físicos}, see={paginacao}}
\newglossaryentry{memoria-virtual}{name={memória virtual}, description={abstração de endereços isolados por processo}, see={mmu}}
\newglossaryentry{paginacao}{name={paginação}, description={divisão da memória em páginas fixas}, see={segmentacao}}
\newglossaryentry{segmentacao}{name={segmentação}, description={memória organizada em segmentos lógicos variáveis}, see={paginacao}}
\newglossaryentry{swapping}{name={swapping}, description={movimentação de páginas/processos inativos para disco}}
\newglossaryentry{dram}{name={DRAM}, description={memória dinâmica principal que exige refresh}, see={sram}}
\newglossaryentry{sram}{name={SRAM}, description={memória estática rápida usada em cache}, see={dram}}
\newglossaryentry{cache}{name={cache}, description={memória rápida que armazena dados frequentes reduzindo latência}, see={sram}}
% Estilo do glossário
\setglossarystyle{list}

% % ---
% % Exemplo de configurações do glossairo
\renewcommand*{\glsseeformat}[3][\seename]{\textit{#1}\glsseelist{#2}}
% % ---


\begin{document}

\pretextual 
\imprimircapa
\imprimirfolhaderosto[]* % talvez dê merda na numeração das páginas
%\cleardoublepage

% ---
% Epígrafe
% ---
\begin{epigrafe}
    \vspace*{\fill}
	\begin{flushright}
		\textit
        {
            "Grandes poderes trazem \\
            grandes responsabilidades." \\
            (Homem-Aranha)
        }
	\end{flushright}
\end{epigrafe}
% ---

\begin{resumo}[Resumo]
    Este trabalho apresenta uma visão sintética dos principais componentes e
    mecanismos de sistemas operacionais modernos, tomando o Linux como
    referência conceitual. Iniciamos pelo processo de boot, contrastando BIOS
    legado e UEFI, destacando o papel do firmware na inicialização do kernel e
    o uso de partições EFI. Em seguida abordamos armazenamento: diferenças
    estruturais entre HDD (trilhas, setores, cilindros) e SSD (páginas, blocos,
    NVMe) e implicações de desempenho. O particionamento é discutido via MBR e
    GPT, incluindo limitações, endereçamento e integridade. Examinamos a camada
    de sistemas de arquivos (inodes, blocos, metadados) e famílias clássicas e
    modernas (Ext, FAT, NTFS, BtrFS, ZFS) com ênfase em journaling e Copy-on-Write.
    Tratamos gestão de usuários e permissões (UID, GID, bits r/w/x), processo e
    segurança via sudo. No gerenciamento de processos descrevemos PID, PCB e
    estados (execução, pronto, bloqueado) e a troca de contexto. Finalizamos com
    técnicas de gerenciamento de memória primária: endereçamento virtual,
    paginação, segmentação e swapping, relacionando isolamento e eficiência. A
    síntese integra camadas de hardware e software para contextualizar decisões
    de projeto que impactam desempenho, segurança e confiabilidade.
    \vspace{\onelineskip}
    \noindent\textbf{Palavras-chave}: sistemas operacionais; boot; UEFI; MBR; GPT; HDD; SSD; sistemas de arquivos; processos; memória; permissões.
\end{resumo}

\begin{resumo}[Abstract]
    \begin{otherlanguage*}{english}
        This report provides a concise overview of core components and
        mechanisms of modern operating systems with Linux as the guiding
        reference. It starts with the boot process, contrasting legacy BIOS and
        UEFI, outlining firmware responsibilities and the role of the EFI
        system partition. Storage is examined through structural differences
        between HDDs (tracks, sectors, cylinders) and SSDs (flash pages, blocks,
        NVMe) and their performance implications. Disk partitioning is discussed
        via MBR and GPT, emphasizing limitations, addressing and integrity.
        The filesystem layer (inodes, blocks, metadata) is reviewed across
        traditional and modern families (Ext, FAT, NTFS, BtrFS, ZFS) focusing on
        journaling and Copy-on-Write. User and permission management (UID, GID,
        r/w/x bits) and privilege elevation through sudo are presented. Process
        management covers PID, PCB, execution states (running, ready, blocked)
        and context switching. Finally, primary memory techniques—virtual
        addressing, paging, segmentation and swapping—are related to isolation
        and efficiency. The synthesis connects hardware and software layers to
        contextualize design choices affecting performance, security and
        reliability.
        \vspace{\onelineskip}
        \noindent\textbf{Keywords}: operating systems; boot; UEFI; MBR; GPT; HDD; SSD; filesystems; processes; memory; permissions.
    \end{otherlanguage*}
\end{resumo}


\pdfbookmark[0]{\listfigurename}{lof}
    \listoffigures*
\cleardoublepage


\pdfbookmark[0]{\listtablename}{lot}
    \listoftables*
\cleardoublepage


% \begin{siglas}
%     \item[sigla] significado da sigla;
% \end{siglas}


% \begin{simbolos}
%     \item[simbolo] descrição;
% \end{simbolos}


\pdfbookmark[0]{\contentsname}{toc}
% Versão sem estrela para garantir exibição do Sumário
    \tableofcontents*
\cleardoublepage


\textual % a partir daqui são os elementos textuais.

% \include{XXX} % adicionar os .tex da pasta inputs
\part{Discos}

\begin{figure}[h!]
    \centering
    \includegraphics[width=0.6\textwidth]{imgs/disk_types.jpg}
    \caption{Tipos de discos de armazenamento.}
\end{figure}

% ------------------------------------------------------------------------
\chapter{Processo de Boot}\label{chap:boot}
% ------------------------------------------------------------------------
O processo de inicialização (\textit{boot}) é a sequência de operações que o computador realiza ao ser ligado para carregar o sistema operacional. O termo deriva de "pull oneself up by one’s bootstraps". Conceitos fundamentais de boot e interação com o kernel são discutidos em \cite{tanenbaum:mos,silberschatz:osc,love:linuxkernel}.

\section{BIOS e ROM}\label{sec:bios}
O processo inicia-se na ROM (\textit{Read-Only Memory}), onde reside o BIOS (\textit{Basic Input/Output System}). O BIOS é gravado em um chip e é responsável pelo teste de hardware (POST) e pela inicialização básica. As configurações do BIOS (sequência de boot, data/hora) são armazenadas em uma memória CMOS alimentada por bateria.
Historicamente, evoluímos de memórias PROM (programáveis uma vez), EPROM (apagáveis por UV), EEPROM (apagáveis eletricamente) até as modernas Flash ROM, que permitem atualizações de firmware \cite{patterson:hennessy}. A padronização moderna de firmware segue a especificação \cite{uefi:spec}.

\section{UEFI e Legacy}\label{sec:uefi}
Nos anos 90, a Intel iniciou a substituição do BIOS pelo EFI, que evoluiu para o padrão UEFI (\textit{Unified Extensible Firmware Interface}) \cite{uefi:spec}.
\begin{itemize}
    \item \textbf{Legacy BIOS:} Utiliza o MBR para boot. Carrega apenas o primeiro estágio do \textit{bootloader} nos primeiros 446 bytes do disco.
    \item \textbf{UEFI:} Armazena na NVRAM o caminho para os carregadores. Utiliza uma partição específica (Partição EFI ou ESP), formatada geralmente em FAT32, onde residem os arquivos \texttt{.efi} dos sistemas operacionais.
\end{itemize}

\section{Gerenciadores de Boot}\label{sec:bootloaders}
O carregador de inicialização (\textit{bootloader}), como o GRUB ou LILO, é carregado pelo firmware. Sua função é carregar o Kernel do sistema operacional na memória RAM e transferir o controle para ele \cite{love:linuxkernel}. No Linux, o Kernel geralmente utiliza um sistema de arquivos temporário (\textit{initramfs}) para montar a partição raiz.

% ------------------------------------------------------------------------
\chapter{Memória Secundária}\label{chap:memoria-secundaria}
% ------------------------------------------------------------------------
A memória secundária, ou memória de massa, é não-volátil e utilizada para armazenar grandes quantidades de dados fora do processador \cite{patterson:hennessy}.

\section{Discos Rígidos (HDD)}\label{sec:hdd}
Dominaram o mercado por décadas. São compostos por discos magnéticos (pratos) e cabeças de leitura/gravação mecânicas.
\begin{itemize}
    \item \textbf{Geometria:} Organizados em Trilhas (círculos concêntricos), Setores (fatias das trilhas, geralmente 512 bytes) e Cilindros (conjunto de trilhas alinhadas verticalmente).
    \item \textbf{Endereçamento:} Antigamente usava-se CHS (Cylinder-Head-Sector). Modernamente utiliza-se LBA (\textit{Logical Block Addressing}), que trata o disco como uma sequência linear de blocos.
\end{itemize}

\section{Unidades de Estado Sólido (SSD)}\label{sec:ssd}
Utilizam memória Flash (NAND ou NOR), baseada em transistores que armazenam carga elétrica (bits) sem partes móveis \cite{nvme:spec}.
\begin{itemize}
    \item \textbf{Vantagens:} Alta velocidade de acesso, resistência a impactos e menor consumo de energia.
    \item \textbf{Estrutura:} Não possuem trilhas ou setores físicos, organizando-se em páginas e blocos. O protocolo NVMe (\textit{Non-Volatile Memory Express}) foi criado para explorar a velocidade do barramento PCIe, superando as limitações do padrão SATA.
\end{itemize}

\begin{figure}[h!]
    \centering
    \begin{tikzpicture}
        \begin{axis}[
            ybar,
            width=0.7\textwidth,
            height=0.45\textwidth,
            bar width=25pt,
            ylabel={Tempo de acesso médio (ms)},
            symbolic x coords={HDD,SSD},
            xtick=data,
            ymin=0,
            ymax=9,
            nodes near coords,
            every node near coord/.append style={font=\small},
            title={Comparação aproximada de latência}
        ]
            \addplot coordinates {(HDD,8) (SSD,0.1)};
        \end{axis}
    \end{tikzpicture}
    \caption{Latência típica: HDD \textasciitilde8 ms vs SSD \textasciitilde0.1 ms.}\label{fig:latencia-hdd-ssd}
\end{figure}

\noindent A redução drástica de latência em SSDs impacta também estratégias de paginação discutidas no Capítulo \ref{chap:memoria}.

% ------------------------------------------------------------------------
\chapter{Partições}\label{chap:particoes}

\begin{figure}[h!]
    \centering
    \includegraphics[width=0.8\textwidth]{imgs/disk_partition.png}
    \caption{Particionamento de disco rígido.}
\end{figure}
% ------------------------------------------------------------------------
O particionamento divide o disco físico em unidades lógicas. Existem dois esquemas principais de tabelas de partição.

\section{MBR (Master Boot Record)}\label{sec:mbr}
O MBR localiza-se no primeiro setor do disco (512 bytes) \cite{gpt:efi}.
\begin{itemize}
    \item \textbf{Estrutura:} Contém a área de boot (446 bytes), a tabela de partições (64 bytes) e a assinatura (2 bytes).
    \item \textbf{Limitações:} Suporta apenas 4 partições primárias. Para contornar isso, criou-se a \textit{Partição Estendida}, que pode conter múltiplas \textit{Partições Lógicas}. Endereça no máximo 2TB de espaço devido ao limite de 32 bits.
\end{itemize}

\section{GPT (GUID Partition Table)}\label{sec:gpt}
Introduzido com o UEFI para superar as limitações do MBR (ver Seção \ref{sec:mbr}) \cite{gpt:efi,uefi:spec}.
\begin{itemize}
    \item \textbf{Características:} Utiliza endereçamento de 64 bits (suporta até 9.4 ZB). Permite um número praticamente ilimitado de partições (padrão de 128 no Windows).
    \item \textbf{Segurança:} Possui backup da tabela de partições no final do disco e CRC32 para verificação de integridade.
\end{itemize}


\begin{figure}[h!]
    \centering
    \begin{tikzpicture}
        \begin{axis}[
            ybar,
            width=0.85\textwidth,
            height=0.45\textwidth,
            bar width=20pt,
            ylabel={Tamanho máximo endereçável (TB)},
            symbolic x coords={MBR,GPT},
            xtick=data,
            ymin=0,
            ymax=10000,
            enlarge x limits=0.25,
            yticklabel style={/pgf/number format/fixed},
            title={Limites teóricos simplificados}
        ]
            % MBR ~2 TB, GPT muito maior (representamos 9000 TB como aproximação simbólica)
            \addplot coordinates {(MBR,2) (GPT,9000)};
        \end{axis}
    \end{tikzpicture}
    \caption{Diferença de capacidade entre MBR e GPT (valores ilustrativos).}\label{fig:capacidade-mbr-gpt}
\end{figure}

\noindent A maior capacidade e redundância da GPT cooperam com recursos avançados de sistemas de arquivos modernos (Capítulo \ref{chap:filesystem}).

\section{Gerenciamento}
Ferramentas como \texttt{fdisk}, \texttt{gdisk} e \texttt{parted} são usadas para criar partições. O comando \texttt{lsblk} lista dispositivos de bloco. Para uso, as partições devem ser formatadas e montadas (\texttt{mount}) em um diretório do sistema. O arquivo \texttt{/etc/fstab} controla as montagens automáticas no boot.

% ------------------------------------------------------------------------
\chapter{Sistema de Arquivos}\label{chap:filesystem}
% ------------------------------------------------------------------------
O Sistema de Arquivos (\textit{Filesystem}) é a estrutura lógica usada para organizar e armazenar dados nas partições. No Linux, o VFS (\textit{Virtual Filesystem}) abstrai os diferentes tipos de sistemas para o Kernel \cite{love:linuxkernel,tanenbaum:mos}.

\section{Conceitos Básicos}
\begin{itemize}
    \item \textbf{Área de Controle vs. Dados:} Metadados (permissões, datas, localização) ficam na área de controle; o conteúdo real fica na área de dados.
    \item \textbf{Inode (Index Node):} Estrutura fundamental no Linux que armazena os metadados de um arquivo. Cada arquivo é identificado por um número de inode único na partição.
    \item \textbf{Blocos:} Unidade mínima de alocação (geralmente 4KB).
\end{itemize}

\section{Famílias de Sistemas de Arquivos}
\begin{itemize}
    \item \textbf{Ext (Ext2, Ext3, Ext4):} Padrão no Linux. O Ext3 introduziu o \textit{journaling} \cite{tweedie:ext3}. O Ext4 suporta volumes de até 1 EB.
    \item \textbf{FAT (FAT16, FAT32, exFAT):} Comuns em pendrives e compatibilidade com Windows. Não suportam permissões POSIX nativamente e sofrem com fragmentação.
    \item \textbf{NTFS:} Padrão do Windows, suporta ACLs, compressão e \textit{journaling}.
    \item \textbf{Sistemas Modernos (CoW):} BtrFS \cite{rodeh:btrfs} e ZFS \cite{bonwick:zfs} utilizam \textit{Copy-on-Write}, snapshots e alta integridade; LFS \cite{rosenblum:lfs} explora escrita sequencial.
\end{itemize}

\begin{table}[h!]
    \centering
    \begin{adjustbox}{width=\textwidth}
        \centering
        \begin{tabular}{|l|c|c|c|c|}
            \hline
            extbf{FS} & \textbf{Journaling} & \textbf{Copy-on-Write} & \textbf{Snapshots}          & \textbf{Referência}  \\
            \hline
            Ext3/Ext4 & Sim                 & Não                    & Limitado (Ext4 via tooling) & \cite{tweedie:ext3}  \\
            BtrFS     & Sim                 & Sim                    & Sim                         & \cite{rodeh:btrfs}   \\
            ZFS       & Sim                 & Sim                    & Sim                         & \cite{bonwick:zfs}   \\
            LFS       & Não                 & Não                    & Não                         & \cite{rosenblum:lfs} \\
            FAT/exFAT & Não                 & Não                    & Não                         & (Especificação)      \\
            NTFS      & Sim                 & Não                    & Shadow copies (Windows)     & (Microsoft Docs)     \\
            \hline
        \end{tabular}
    \end{adjustbox}
    \caption{Resumo de características de sistemas de arquivos.}
    \label{tab:fs-caracteristicas}
\end{table}

\noindent A escolha do sistema de arquivos impacta recuperação e integridade (ver journaling \cite{tweedie:ext3}).

\section{Links}
\begin{itemize}
    \item \textbf{Hard Link:} Um novo nome para o mesmo inode. Só funciona na mesma partição e não se aplica a diretórios. Se o original for apagado, o dado persiste.
    \item \textbf{Soft Link (Simbólico):} Um arquivo especial que aponta para o caminho de outro arquivo. Pode cruzar partições. Se o alvo for apagado, o link quebra.
\end{itemize}
\part{Permissões}

% ------------------------------------------------------------------------
\chapter{Grupos, Usuários, Poderes e Permissões}\label{chap:permissoes}
% ------------------------------------------------------------------------
\begin{figure}[h!]
    \centering
    \includegraphics[width=0.8\textwidth]{imgs/linux_permissions.png}
    \caption{Visão geral de usuários, grupos e permissões no Linux.}
\end{figure}

O Linux é um sistema multiusuário que controla o acesso através de UIDs (User IDs) e GIDs (Group IDs) \cite{love:linuxkernel,tanenbaum:mos}.

\section{Gerenciamento de Usuários e Grupos}\label{sec:usuarios-grupos}
\begin{itemize}
    \item \textbf{Arquivos de Configuração:}
    \begin{itemize}
        \item \texttt{/etc/passwd}: Mapeia usuários para UIDs, shells e diretórios home.
        \item \texttt{/etc/shadow}: Armazena as senhas criptografadas (hashes) de forma segura.
        \item \texttt{/etc/group}: Define os grupos e seus membros.
    \end{itemize}
    \item \textbf{Comandos:} \texttt{adduser}, \texttt{deluser}, \texttt{passwd}, \texttt{addgroup}.
\end{itemize}

\begin{figure}[h!]
    \centering
    \includegraphics[width=0.8\textwidth]{imgs/user_data.png}
    \caption{Propriedades do usuário no Linux.}
\end{figure}

\section{Permissões de Arquivos (Modo Octal e Simbólico)}\label{sec:permissoes-arquivos}
Cada arquivo possui permissões para Dono (u), Grupo (g) e Outros (o) \cite{silberschatz:osc}. A estrutura de metadados de cada arquivo (inode) é detalhada no Capítulo \ref{chap:filesystem}.
\begin{itemize}
    \item \textbf{Tipos:} Leitura (\textbf{r}=4), Escrita (\textbf{w}=2), Execução (\textbf{x}=1).
    \item \textbf{Comandos:}
    \begin{itemize}
        \item \texttt{chmod}: Altera as permissões (ex: \texttt{chmod 755 arquivo}).\footnote{Manual: \url{https://man7.org/linux/man-pages/man1/chmod.1.html}}
        \item \texttt{chown}: Altera o dono e o grupo (ex: \texttt{chown user:group arquivo}).\footnote{Manual: \url{https://man7.org/linux/man-pages/man1/chown.1.html}}
    \end{itemize}
\end{itemize}

\section{Privilégios Especiais}\label{sec:privilegios-especiais}
O usuário \texttt{root} (UID 0) tem acesso total. O comando \texttt{sudo} (\textit{SuperUser Do}) permite que usuários comuns executem comandos com privilégios elevados, configurados via \texttt{/etc/sudoers} \cite{stallings:os}. Essa elevação é frequentemente necessária para operações de montagem de partições (ver Capítulo \ref{chap:particoes}).

% ------------------------------------------------------------------------
\chapter{Gerenciamento de Processos}\label{chap:processos}
% ------------------------------------------------------------------------
Um processo é uma instância de um programa em execução. O gerenciamento de processos é central para o sistema operacional, permitindo multitarefa através de \textit{time sharing} \cite{tanenbaum:mos,silberschatz:osc}. Processos essenciais são disparados já na fase de inicialização (Capítulo \ref{chap:boot}).

\section{Identificação e Estrutura}\label{sec:identificacao-processo}
\begin{itemize}
    \item \textbf{PID:} Todo processo possui um identificador único (\textit{Process ID}).
    \item \textbf{PCB (Process Control Block):} Estrutura de dados no Kernel que armazena o contexto do processo (registradores, estado, prioridade, contadores).
    \item \textbf{Contexto:} Hardware (registradores da CPU), Software (quotas, privilégios) e Endereçamento (memória alocada).
\end{itemize}

\section{Estados do Processo}\label{sec:estados-processo}
Um processo transita entre estados:
\begin{enumerate}
    \item \textbf{Execução (Running):} Está usando a CPU.
    \item \textbf{Pronto (Ready):} Aguardando vez na CPU (escalonamento).
    \item \textbf{Espera (Wait/Blocked):} Aguardando um evento (ex: E/S).
\end{enumerate}
A troca entre processos é chamada de \textit{Mudança de Contexto} \cite{love:linuxkernel}.

\begin{table}[h!]
    \centering
    \begin{adjustbox}{width=\textwidth}
        \begin{tabular}{|l|l|l|}
            \hline
            \textbf{Estado} & \textbf{Descrição resumida}   & \textbf{Referência}     \\
            \hline
            Running       & Em execução na CPU            & \cite{tanenbaum:mos}    \\
            Ready         & Apto, aguardando CPU          & \cite{silberschatz:osc} \\
            Blocked       & Esperando evento E/S          & \cite{stallings:os}     \\
            Zombie        & Finalizado, aguardando coleta & \cite{love:linuxkernel} \\
            Stopped       & Suspenso por sinal            & \cite{love:linuxkernel} \\
            \hline
        \end{tabular}
    \end{adjustbox}
    \caption{Estados de processo e descrição.}\label{tab:estados-processo}
\end{table}

\section{Monitoramento}\label{sec:monitoramento-processo}
O sistema de arquivos virtual \texttt{/proc} expõe informações do Kernel sobre processos. Comandos como \texttt{top}, \texttt{htop}, \texttt{ps} e \texttt{uptime} (para ver média de carga/load average) são usados para monitoramento \cite{love:linuxkernel}. A interação com \texttt{/proc} complementa a análise de desempenho de discos (Capítulo \ref{chap:memoria-secundaria}).\footnote{Documentação: \url{https://man7.org/linux/man-pages/man5/proc.5.html}}

% ------------------------------------------------------------------------
\chapter{Gerenciamento de Memória Primária}\label{chap:memoria}
% ------------------------------------------------------------------------
O gerenciador de memória controla o acesso da CPU à memória principal (RAM), alocando e desalocando espaços para processos.

\section{Tipos de Memória}\label{sec:tipos-memoria}
\begin{itemize}
    \item \textbf{RAM (Random Access Memory):} Memória volátil de acesso rápido. Divide-se em DRAM (Dinâmica, precisa de refresh, usada na memória principal) e SRAM (Estática, mais rápida, usada em Cache L1/L2).
    \item \textbf{Cache:} Armazena dados usados frequentemente para reduzir o tempo de acesso à DRAM.
\end{itemize}

\begin{figure}[h!]
    \centering
    \begin{tikzpicture}
        \begin{axis}[
            ybar,
            width=0.85\textwidth,
            height=0.48\textwidth,
            bar width=16pt,
            ylabel={Latência aproximada (ns)},
            symbolic x coords={Cache L1,DRAM,SSD,HDD},
            xtick=data,
            ymode=log,
            log basis y={10},
            ymin=0.8,
            ymax=10000000,
            nodes near coords,
            nodes near coords align={vertical},
            every node near coord/.append style={font=\scriptsize},
            title={Hierarquia de latências (escala log)}
        ]
            \addplot coordinates {(Cache L1,1) (DRAM,50) (SSD,100000) (HDD,5000000)};
        \end{axis}
    \end{tikzpicture}
    \caption{Ordem de grandeza de latências de memória e armazenamento.}\label{fig:latencias-hierarquia}
\end{figure}

\noindent Valores ilustrativos baseados em ordens de grandeza.\footnote{Referência de latências: \url{https://people.eecs.berkeley.edu/~rcs/research/interactive_latency.html}} Ver Capítulo \ref{chap:memoria-secundaria} para detalhes físicos de SSD e HDD \cite{patterson:hennessy}.

\section{Técnicas de Gerenciamento}\label{sec:tecnicas-gerenciamento}
\begin{itemize}
    \item \textbf{Endereçamento Virtual:} Os processos usam endereços lógicos que são traduzidos para endereços físicos pela MMU (Memory Management Unit). Isso isola a memória de cada processo \cite{tanenbaum:mos,silberschatz:osc}.
    \item \textbf{Paginação:} Divide a memória em blocos de tamanho fixo chamados páginas (geralmente 4KB). Permite carregar partes do processo sob demanda e elimina fragmentação externa \cite{stallings:os}.
    \item \textbf{Segmentação:} Divide a memória em segmentos de tamanhos variáveis baseados na lógica do programa (código, dados, pilha) \cite{silberschatz:osc}.
    \item \textbf{Swapping:} Quando a RAM está cheia, o sistema move processos inativos para uma área no disco (Swap), liberando memória principal. O acesso ao disco é significativamente mais lento que à RAM \cite{patterson:hennessy}.
\end{itemize}

\begin{table}[h!]
    \centering
    \begin{adjustbox}{width=\textwidth}
        \centering
        \begin{tabular}{|l|l|l|l|}
            \hline
            \textbf{Permissão} & \textbf{Valor} & \textbf{Descrição}           & \textbf{Referência}     \\
            \hline
            r                & 4              & Leitura                      & \cite{silberschatz:osc} \\
            w                & 2              & Escrita                      & \cite{stallings:os}     \\
            x                & 1              & Execução                     & \cite{love:linuxkernel} \\
            rwx              & 7              & Leitura + Escrita + Execução & \cite{tanenbaum:mos}    \\
            rw-              & 6              & Leitura + Escrita            & (POSIX)                 \\
            r-x              & 5              & Leitura + Execução           & (POSIX)                 \\
            ---              & 0              & Sem acesso                   & (POSIX)                 \\
            \hline
        \end{tabular}
    \end{adjustbox}
    \caption{Mapa de permissões simbólicas e octais.}\label{tab:permissoes-octal}
\end{table}
\part{Exercícios}

\chapter{Atividades - Capítulo 6}

\section{Atividade 6.1}
\subsection*{Enunciado}
Pesquise as opções do comando \texttt{ps} e crie um comando para listar apenas os identificadores e nomes dos processos que estão no estado \textbf{R} (running). Verifique também o comando \texttt{top}, uma versão interativa do comando \texttt{ps}, atualizando a listagem de processos a cada \textit{n} segundos e ordenando-os por uso de CPU e memória.
Consulte os estados de processo descritos em \ref{sec:estados-processo} e relacione diferenças de carga com operações de disco (Capítulo \ref{chap:memoria-secundaria}).

\section{Atividade 6.2}
\subsection*{Enunciado}
A partir do Shell, execute um programa em background (por exemplo, o navegador web). Observe a árvore hierárquica de processos e visualize as dependências destes processos. Observe ainda seu PID e PPID e compare aos do \texttt{bash}. Verifique também o comando \texttt{jobs}, que mostra a situação de todos os processos em background naquele Shell.

\section{Atividade 6.3}
\subsection*{Enunciado}
Utilize o daemon \texttt{cron} para realizar um backup daqui a cinco minutos dos arquivos \texttt{.log} da pasta \texttt{/var/log}. O arquivo de backup deve ser compactado e colocado dentro de sua pasta.

\begin{tcolorbox}[colback=yellow!10!white,colframe=red!75!black]
Utilize \texttt{crontab -e} para editar o arquivo e o comando \texttt{tar -zcvf <destino> <origem>} para fazer a compactação dos arquivos \texttt{.log}.
\end{tcolorbox}

\section{Atividade 6.4}
\subsection*{Enunciado}
Crie um script que mostra o usuário atual, o diretório atual e o Shell em uso. Lembre que em \texttt{/etc/passwd} encontram-se listados os usuários e seus shells.
Ver também gerenciamento de usuários na Seção \ref{sec:usuarios-grupos}.

\begin{tcolorbox}[colback=yellow!10!white,colframe=red!75!black]
Use o comando \texttt{cut} com a sintaxe:
\begin{lstlisting}[language=bash]
cut -d DELIMITADOR -f NUMERO\_DO\_CAMPO
\end{lstlisting}
O comando \texttt{cut} corta somente o(s) campo(s) de número \texttt{NUMERO\_DO\_CAMPO} e utiliza o separador \texttt{DELIMITADOR} para delimitar cada campo. No caso do arquivo \texttt{passwd}, o delimitador é ":".
\end{tcolorbox}


\section{Atividade 6.8}
\subsection*{Enunciado}
Crie um script que verifica o número de parâmetros recebidos na linha de comando. Verifique se existe apenas um parâmetro. Use o comando \texttt{test} para validar e \texttt{exit} para sair se o número for incorreto. Lembre-se de que o número de parâmetros está na variável \texttt{\$\#}.
Relacione com o contexto de processo (Seção \ref{sec:identificacao-processo}).

\section{Atividade 6.9}
\subsection*{Enunciado}
Estenda a atividade anterior para verificar se o parâmetro é um diretório. Exemplo de uso: \texttt{./testedir /home}. Use o teste \texttt{[[ -d \$1 ]]} para verificar. A variável \texttt{\$?} indica status: zero sucesso; diferente de zero falha.
Considere permissões de acesso (Seção \ref{sec:permissoes-arquivos}).

\section{Atividade 6.10}
\subsection*{Enunciado}
Adapte a atividade anterior para receber dois parâmetros: um diretório e uma letra. Se o primeiro for diretório válido, liste todos os arquivos que começam com a letra informada (ex.: \texttt{ls \$2*}).
Impacto de listar muitos arquivos pode variar conforme latências de armazenamento (Figura \ref{fig:latencia-hdd-ssd}).

\chapter{Atividades - Capítulo 7}

\section{Atividade 7.6}
\subsection*{Enunciado}
Crie um arquivo chamado \texttt{agenda} contendo nomes e telefones (\texttt{nome sobrenome telefone}). Crie um script que recebe como parâmetro um nome e imprime as linhas do arquivo \texttt{agenda} que o contenham. Verifique se foi passado parâmetro.

    \begin{lstlisting}[language=bash]
        #!/bin/bash
        ???????????????????

        then
            ???????????????????
            ???????????????????
            ???????????????????

            exit 1
        else
            ???????????????????
            ???????????????????
            then    
                echo 'Recuperando informacoes'
                grep ?????????? agenda
            fi
        fi
    \end{lstlisting}

\section{Atividade 7.9}
\subsection*{Enunciado}
1. Crie um script que recebe duas palavras como parâmetro e verifica se a primeira está contida na segunda.\\ 
2. Crie um script que recebe o nome de um arquivo texto como parâmetro, verifica se o usuário possui permissão de leitura deste arquivo e, caso tenha, 
exiba as seguintes informações sobre o arquivo: número de caracteres, número de palavras e número de linhas do arquivo \texttt{(utilize o comando wc)}.

\section{Atividade 7.11}
    \subsection*{Enunciado}
    Crie um script que apresenta um menu:
    \begin{enumerate}
        \item Exibir status das partições (\texttt{df -h})
        \item Exibir usuários logados (\texttt{who})
        \item Exibir data/hora (\texttt{date})
        \item Sair
    \end{enumerate}
    Leia a opção e execute o comando correspondente; valide entradas inválidas.

\postextual % a partir daqui são os elementos pós-textuais.

% Removido nocite genérico; citações agora inseridas diretamente no texto.

\bibliographystyle{abntex2-alf} % estilo abntex2 de referências bibliográficas
\bibliography{ref} % inclui o arquivo ref.bib

% garantir que todas as entradas do glossário sejam impressas, mesmo sem citações
\glsaddall
\addcontentsline{toc}{chapter}{\glossaryname}
\printglossaries

% \chapter{APÊNDICE}
% \chapter{anexo}
\end{document}